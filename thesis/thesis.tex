\documentclass[12pt,a4paper]{report}
\setlength\textwidth{145mm}
\setlength\textheight{247mm}
\setlength\oddsidemargin{15mm}
\setlength\evensidemargin{15mm}
\setlength\topmargin{0mm}
\setlength\headsep{0mm}
\setlength\headheight{0mm}

\let\openright=\clearpage

\usepackage[utf8]{inputenc}

\usepackage{graphicx}
\usepackage{amsthm}

\usepackage[ps2pdf,unicode]{hyperref}
\hypersetup{pdftitle=Scala Web Application Toolkit}
\hypersetup{pdfauthor=Jan �irok�}

\makeatletter
\def\@makechapterhead#1{
  {\parindent \z@ \raggedright \normalfont
   \Huge\bfseries \thechapter. #1
   \par\nobreak
   \vskip 20\p@
}}
\def\@makeschapterhead#1{
  {\parindent \z@ \raggedright \normalfont
   \Huge\bfseries #1
   \par\nobreak
   \vskip 20\p@
}}
\makeatother

\def\chapwithtoc#1{
\chapter*{#1}
\addcontentsline{toc}{chapter}{#1}
}

\begin{document}

\lefthyphenmin=2
\righthyphenmin=2

%%% Front Page

\pagestyle{empty}
\begin{center}

\large

Charles University in Prague

\medskip

Faculty of Mathematics and Physics

\vfill

{\bf\Large MASTER THESIS}

\vfill

\centerline{\mbox{\includegraphics[width=60mm]{logo.eps}}}

\vfill
\vspace{5mm}

{\LARGE Jan �irok�}

\vspace{15mm}

{\LARGE\bfseries Scala Web Application Toolkit}

\vfill

Department of Distributed and Dependable Systems
\vfill

\begin{tabular}{rl}

Supervisor of the master thesis: & Pavel Je�ek \\
\noalign{\vspace{2mm}}
Study programme: & Computer Science \\
\noalign{\vspace{2mm}}
Specialization: & Software Systems \\
\end{tabular}

\vfill

Prague 2013

\end{center}

\newpage

%%% Následuje vevázaný list -- kopie podepsaného "Zadání diplomové práce".
%%% Toto zadání NENÍ součástí elektronické verze práce, nescanovat.

%%% Na tomto místě mohou být napsána případná poděkování (vedoucímu práce,
%%% konzultantovi, tomu, kdo zapůjčil software, literaturu apod.)
\openright

\noindent
Dedication.

\newpage

%%% Strana s čestným prohlášením k diplomové práci

\vglue 0pt plus 1fill

\noindent
I declare that I carried out this master thesis independently, and only with the cited
sources, literature and other professional sources.

\medskip\noindent
I understand that my work relates to the rights and obligations under the Act No.
121/2000 Coll., the Copyright Act, as amended, in particular the fact that the Charles
University in Prague has the right to conclude a license agreement on the use of this
work as a school work pursuant to Section 60 paragraph 1 of the Copyright Act.

\vspace{10mm}

\hbox{\hbox to 0.5\hsize{%
In ........ date ............
\hss}\hbox to 0.5\hsize{%
signature of the author
\hss}}

\vspace{20mm}
\newpage

%%% Povinná informační strana diplomové práce

\vbox to 0.5\vsize{
\setlength\parindent{0mm}
\setlength\parskip{5mm}

Název práce:
Název práce
% přesně dle zadání

Autor:
Jméno a příjmení autora

Katedra:  % Případně Ústav:
Název katedry či ústavu, kde byla práce oficiálně zadána
% dle Organizační struktury MFF UK

Vedoucí diplomové práce:
Jméno a příjmení s tituly, pracoviště
% dle Organizační struktury MFF UK, případně plný název pracoviště mimo MFF UK

Abstrakt:
% abstrakt v rozsahu 80-200 slov; nejedná se však o opis zadání diplomové práce

Klíčová slova:
% 3 až 5 klíčových slov

\vss}\nobreak\vbox to 0.49\vsize{
\setlength\parindent{0mm}
\setlength\parskip{5mm}

Title:
% přesný překlad názvu práce v angličtině

Author:
Jméno a příjmení autora

Department:
Název katedry či ústavu, kde byla práce oficiálně zadána
% dle Organizační struktury MFF UK v angličtině

Supervisor:
Jméno a příjmení s tituly, pracoviště
% dle Organizační struktury MFF UK, případně plný název pracoviště
% mimo MFF UK v angličtině

Abstract:
% abstrakt v rozsahu 80-200 slov v angličtině; nejedná se však o překlad
% zadání diplomové práce

Keywords:
% 3 až 5 klíčových slov v angličtině

\vss}

\newpage

%%% Strana s automaticky generovaným obsahem diplomové práce. U matematických
%%% prací je přípustné, aby seznam tabulek a zkratek, existují-li, byl umístěn
%%% na začátku práce, místo na jejím konci.

\openright
\pagestyle{plain}
\setcounter{page}{1}
\tableofcontents

%%% Jednotlivé kapitoly práce jsou pro přehlednost uloženy v samostatných souborech
\include{preface}
\include{chap1}
\include{chap2}

% Ukázka použití některých konstrukcí LateXu (odkomentujte, chcete-li)
% \include{example}

\include{epilog}

%%% Seznam použité literatury
\begin{thebibliography}{99}
\addcontentsline{toc}{chapter}{\bibname}

\bibitem{Flash}
	Introducing the Adobe Flash Platform.\\
	http://www.adobe.com/devnet/flashplatform/articles/flashplatform\_overview.html
	
\bibitem{JavaApplets}
	Java applet.\\
	http://en.wikipedia.org/wiki/Java\_applet
	
\bibitem{JavaScript}
	JavaScript.\\
	http://en.wikipedia.org/wiki/JavaScript
	
\bibitem{EcmaScript}
	{\sc Ecma} International.\\
	ECMAScript Language Specification.\\
	http://www.ecma-international.org/publications/files/ECMA-ST/Ecma-262.pdf
	
\bibitem{Ajax}
	{\sc Garrett,} Jesse James.\\
	Ajax: A New Approach to Web Applications.\\
	http://www.adaptivepath.com/ideas/ajax-new-approach-web-applications

\bibitem{Html5}
	HTML5 Introduction.\\
	http://www.w3schools.com/html/html5\_intro.asp

\bibitem{Backends}
  {\sc Ashkenas,} Jeremy.\\
  List of languages that compile to JS.\\
  https://github.com/jashkenas/coffee-script/wiki/List-of-languages-that-compile-to-JS
	
\bibitem{Meijer}
	{\sc Meijer,} Erik and {\sc Drayton,} Peter.\\
	Static Typing Where Possible, Dynamic Typing When Needed: The End of the Cold War Between Programming Languages.\\
	http://research.microsoft.com/en-us/um/people/emeijer/Papers/RDL04Meijer.pdf
	
\bibitem{RequireJs}
	RequireJS.\\
	http://requirejs.org/
	
\bibitem{HeadJs}
	HeadJS.\\
	http://headjs.com/
	
\bibitem{JsLint}
	JSLint, The JavaScript Code Quality Tool.\\
	http://www.jslint.com/
	
\bibitem{JsHint}
	JSHint, a JavaScript Code Quality Tool.\\
	http://www.jshint.com/

\bibitem{GoogleClosure}
	Google Closure Compiler.\\
	http://closure-compiler.appspot.com/home
	
\bibitem{Dart}
	Dart: Structured web apps.\\
	www.dartlang.org/
	
\bibitem{CoffeeScript}
	CoffeeScript.\\
	http://coffeescript.org/
	
\bibitem{TypeScript}
	TypeScript.\\
	www.typescriptlang.org/
	
\bibitem{NodeJs}
	node.js\\
	http://nodejs.org/
	
\bibitem{Rpc}
	Remote procedure call.\\
	http://en.wikipedia.org/wiki/Remote\_procedure\_call
	
\bibitem{Gwt}
	GWT Project.\\
	www.gwtproject.org/
	
\bibitem{SharpKit}
	SharpKit - C\# to JavaScript Compiler.\\
	http://sharpkit.net/
	
\bibitem{Tiobe}
	TIOBE Programming Community Index for July 2013.\\
	http://www.tiobe.com/index.php/content/paperinfo/tpci/index.html
	
\bibitem{Dom}
	Document Object Model (DOM) Specifications.\\
	http://www.w3.org/DOM/DOMTR
		
\bibitem{Jquery}
	jQuery.\\
	http://jquery.com/
	
\bibitem{Sbt}
	Scala Build Tool.\\
	http://www.scala-sbt.org/
	
\bibitem{Play}
	Play.\\
	http://www.playframework.com/
	
\bibitem{Typesafe}
	Typesafe.\\
	http://typesafe.com/
	
\bibitem{Self}
	Self.
	http://selflanguage.org/
	
\bibitem{ScalableComponents}
	{\sc Odersky,} Martin and {\sc Zenger,} Matthias.\\
	Scalable Component Abstractions.\\
	http://lampwww.epfl.ch/~odersky/papers/ScalableComponent.pdf
	
\bibitem{Linq}
  LINQ (Language-Integrated Query).\\
  http://msdn.microsoft.com/en-us/library/vstudio/bb397926.aspx
	
\bibitem{ScalaProgramming}
	{\sc Odersky,} Martin and {\sc Spoon,} Lex and {\sc Venners,} Bill.\\  Programming in Scala, Second Edition.\\
	Artima, 2010
	
\bibitem{ScalaAdvancedTypes}
	Advantages of Scala's Type System.\\
	http://stackoverflow.com/questions/3112725/advantages-of-scalas-type-system
	
\bibitem{Linearization}
  {\sc McBeath}, Jim.\\
	Scala Class Linearization.\\
	http://jim-mcbeath.blogspot.cz/2009/08/scala-class-linearization.html
	
\bibitem{Reflection}
	Symbols, Trees, and Types.\\
	http://docs.scala-lang.org/overviews/reflection/symbols-trees-types.html
	
\bibitem{SourceMaps}
  {\sc Seddon,} Ryan.\\
	Introduction to JavaScript Source Maps.\\
  http://www.html5rocks.com/en/tutorials/developertools/sourcemaps/
	
\bibitem{W3c}
  World Wide Web Consortium.\\
	http://www.w3.org/
	
\bibitem{ScalaDynamic}
  Scala Dynamic.\\
	http://www.scala-lang.org/api/current/index.html\#scala.Dynamic
	
\bibitem{ScalaLibrary}
	Scala Standard Library API.\\
  http://www.scala-lang.org/api/current/index.html\#package
	
\bibitem{Lazy}
  Lazy initialization.\\
	http://en.wikipedia.org/wiki/Lazy\_initialization
	
\bibitem{CompilerPlugins}
  Writing Scala Compiler Plugins.\\
  http://www.scala-lang.org/node/140
	
\bibitem{ScalaFutures}
  Futures and Promises.\\
  http://docs.scala-lang.org/overviews/core/futures.html
	
\bibitem{ScalaGwt}
  Scala+GWT Project.\\
  https://github.com/scalagwt
	
\bibitem{S2js}
  S2JS Compiler Plugin.\\
  https://github.com/alvaroc1/s2js
	
\bibitem{Scalosure}
  Scalosure.\\
  https://github.com/efleming969/scalosure
	
\bibitem{JsScala}
  js.scala: JavaScript as an embedded DSL in Scala.\\
  https://github.com/js-scala/js-scala
	
\bibitem{ScalaJs}
  Scala.js, a Scala to JavaScript compiler.\\
  https://github.com/sjrd/scala-js
	
\bibitem{WebWorkers}
  Using web workers.\\
  https://developer.mozilla.org/en-US/docs/Web/Guide/Performance/Using\_web\_workers
	
\bibitem{Actors}
  Actor Model.\\
  http://en.wikipedia.org/wiki/Actor\_model
	
\bibitem{Akka}
	akka.\\
	http://akka.io/
	
\bibitem{WebSockets}
  Introducing WebSockets: Bringing Sockets to the Web.\\
	http://www.html5rocks.com/en/tutorials/websockets/basics/
	
\bibitem{StringInterpolation}
  Scala Documentation.\\
	String Interpolation.\\
	http://docs.scala-lang.org/overviews/core/string-interpolation.html
	
\bibitem{Equality}
  {\sc Odersky,} Martin and {\sc Spoon,} Lex and {\sc Venners,} Bill.\\
  How to Write an Equality Method in Java.\\
	{sc}
	http://www.artima.com/lejava/articles/equality.html
	
\end{thebibliography}

%%% Tabulky v diplomové práci, existují-li.
\chapwithtoc{List of Tables}

%%% Použité zkratky v diplomové práci, existují-li, včetně jejich vysvětlení.
\chapwithtoc{List of Abbreviations}

%%% Přílohy k diplomové práci, existují-li (různé dodatky jako výpisy programů,
%%% diagramy apod.). Každá příloha musí být alespoň jednou odkazována z vlastního
%%% textu práce. Přílohy se číslují.
\chapwithtoc{Attachments}

\openright
\end{document}
