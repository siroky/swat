\documentclass[12pt,a4paper]{report}
\setlength\textwidth{145mm}
\setlength\textheight{247mm}
\setlength\oddsidemargin{15mm}
\setlength\evensidemargin{15mm}
\setlength\topmargin{0mm}
\setlength\headsep{0mm}
\setlength\headheight{0mm}

\usepackage[utf8]{inputenc}
\usepackage{graphicx}
\usepackage{epstopdf}
\usepackage{amsthm}
\usepackage{nomencl}

\makenomenclature


%%% Style adjustment macros.

% Less indented chapter titles.
\makeatletter
\def\@makechapterhead#1{
  {\parindent \z@ \raggedright \normalfont
   \Huge\bfseries \thechapter. #1
   \par\nobreak
   \vskip 20\p@
}}
\def\@makeschapterhead#1{
  {\parindent \z@ \raggedright \normalfont
   \Huge\bfseries #1
   \par\nobreak
   \vskip 20\p@
}}
\makeatother

% Chapter without a number but still in the table of contents.
\def\chapwithtoc#1{
\chapter*{#1}
\addcontentsline{toc}{chapter}{#1}
}



%%% Document
\begin{document}

\lefthyphenmin=2
\righthyphenmin=2



%%% Title page
\pagestyle{empty}

\begin{center}
\large Charles University in Prague
\medskip

Faculty of Mathematics and Physics

\vfill

{\bf\Large MASTER THESIS}

\vfill

\centerline{\mbox{\includegraphics[width=60mm]{img/logo.eps}}}

\vfill

\vspace{5mm}

{\LARGE Jan Široký}

\vspace{15mm}

{\LARGE\bfseries Scala Web Application Toolkit}

\vfill

Department of Distributed and Dependable Systems

\vfill

\begin{tabular}{rl}
Supervisor of the master thesis: & Pavel Ježek \\
\noalign{\vspace{2mm}}
Study programme: & Computer Science \\
\noalign{\vspace{2mm}}
Specialization: & Software Systems \\
\end{tabular}

\vfill

Prague 2013
\end{center}



%%% Dedication.
\newpage
\noindent 
Dedication.



%%% Declaration.
\newpage

\vglue 0pt plus 1fill
\noindent
I declare that I carried out this master thesis independently, and only with the cited sources, literature and other professional sources.

\medskip\noindent
I understand that my work relates to the rights and obligations under the Act No. 121/2000 Coll., the Copyright Act, as amended, in particular the fact that the Charles University in Prague has the right to conclude a license agreement on the use of this work as a school work pursuant to Section 60 paragraph 1 of the Copyright Act.

\vspace{10mm}
\hbox{\hbox to 0.5\hsize{%
In ........ date ............
\hss}\hbox to 0.5\hsize{%
signature of the author
\hss}}
\vspace{20mm}



%%% Information page.
\newpage

\vbox to 0.5\vsize{
\setlength\parindent{0mm}
\setlength\parskip{5mm}

Název práce: 
Scala Web Application Toolkit

Autor: 
Jan Široký

Katedra: 
Katedra distribuovaných a spolehlivých systémů

Vedoucí diplomové práce: 
Mgr. Pavel Ježek Ph.D, Katedra distribuovaných a spolehlivých systémů

Abstrakt:
TODO

Klíčová slova:
Scala, JavaScript, RIA, JSON, RPC

\vss}\nobreak\vbox to 0.49\vsize{
\setlength\parindent{0mm}
\setlength\parskip{5mm}

Title:
Scala Web Application Toolkit

Author:
Jan Široký

Department:
Department of Distributed and Dependable Systems

Supervisor:
Mgr. Pavel Ježek Ph.D, Department of Distributed and Dependable Systems

Abstract:
TODO

Keywords:
Scala, JavaScript, RIA, JSON, RPC

\vss}



%%% Generated table of contents.
\newpage
\pagestyle{plain}
\setcounter{page}{1}
\tableofcontents



%%% Introduction.
\chapter{Introduction}

Current state of the rich web application development, single page applications. Tendency in past couple of years to compile into JavaScript. JavaScript starts to become the assembler of the web.

\section{Motivation}

JavaScript disadvantages, some examples where type-safety is an advantage over dynamic typing. OOP. RPC. Code sharing.
\nomenclature{OOP}{object oriented programming}

\section{Goals}

Assignment, outline of the following chapters.


%%% Background.
\chapter{Background}

Why needed and purpose. 

\section{JavaScript}

Brief overview of the language, differences from conventional languages (C, Java, C\#), class-based vs prototype based inheritance, scoping.

\section{Scala}

Brief overview of the language with accent on where it's better than conventional languages. Differences from them and from JavaScript.

\subsection{Compiler Architecture}

https://wiki.scala-lang.org/display/SIW/Compiler


%%% Compiler.
\chapter{Swat Compiler}

Architecture, front end backend. Image.

\section{Backend}

\subsection{JavaScript ASTs}

\section{Frontend}

Compile time vs runtime decision.

\subsection{Package Scope}

Compilation on the package level, splitting into classfiles.

\subsection{Type Scope}

Four types of classes, compilation of the members, methods, overloading.

\subsection{Expression Scope}

Everything is an expression, mainly the interesting cases.

\subsection{Build Process}

Integration to SBT, compiler plugin vs. standalone compiler.



%%% Adapters
\chapter{Adapters}

A list of adapters, links to the specifications, walkthrough on how to create them without the necessity to blindly copy the specification using traits as shared interfaces.



%%% Runtime
\chapter{Runtime}

How it works, outline of the following sections.

\section{Client}

Description of everything intereseting in the swat.js: inheritance, method invocation, type system.

\section{Standard Libraries}

\subsection{Java Library}

\subsection{Scala Library}

\section{Common}

Code sharing.

\subsection{Type Loader}

\subsection{JSON Serializer}

\subsection{RPC Dispatcher}

\section{Integration with a Web Framework}



%%% Sample application.
\chapter{Sample Application}

TODO. A nice example could be online SWAT compiler, that would let the user write some Scala code to a textarea, invoke the Swat compiler (through RPC on the server) and display the compiler code.



%%% Conclusion.
\chapter{Conclusion}
% \addcontentsline{toc}{chapter}{Conclusion}

\section{Critical Evaluation}

What was done, what lacks the most (Scala library is the expectation numero uno), but stating that everything got done to some extent. Hopefully.

\section{Comparison with Similar Tools}

The tools at the time of thesis start (ScalaGWT, s2js, scalosure) and current scala-js, js-scala.

\section{Roadmap}

Actually the main purpose and goal is to really simplify tasks, that aren't programmer friendly in JavaScript.

\subsection{Web workers}

Actor like abstraction over web workers, with help of JSON serializer and class-loader.

\subsection{Template engine}

Problems with JavaScript templating and how it could be solved using string interpolation.

\subsection{Build process}

Swat as a SBT plugin.



%%% Bibliography.
\def\bibname{Bibliography}
\begin{thebibliography}{99}
\addcontentsline{toc}{chapter}{\bibname}

%\bibitem{lamport94}
%  {\sc Lamport,} Leslie.
%  \emph{\LaTeX: A Document Preparation System}.
%  2. vydání.
%  Massachusetts: Addison Wesley, 1994.
%  ISBN 0-201-52983-1.

\end{thebibliography}



%%% Tables.
\chapwithtoc{List of Tables}



%%% Abbreviations.
\chapwithtoc{List of Abbreviations}

\printnomenclature



%%% Attachments.
\chapwithtoc{Attachments}


\end{document}
